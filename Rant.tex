\documentclass[letterpaper,12pt]{article}
\usepackage{ifpdf}
\usepackage{mla}
\begin{document}
\begin{mla}{Eric}{Ye}{ENG 3U}{Ms. Speirs}{22 January, 2013}{Numbers, and Numbers, and Numbers}

Every high school student you ask, possibly with the exception of our program, will tell you that calculus is useless. And they have a point. Almost none of the jobs outside Science, Technology, Engineering and Math require calculus, and even within, to use calculus on a daily basis is rare.

Unlike English, which teaches you to be eloquent; or science, which brings us closer to discovering how the world works, calculus is incredibly abstract and impractical. Its so-called ``real-world" applications are repetitive and highly specialized: they amount to understanding the growth of spherical cows in vacuums. And while that might be fun and simple, it is far detached from reality and practically useless.

How about statistics? What real-life use does statistics have?
Plenty. Statistical literacy is surprisingly valuable as a personal asset. The 15 top-earning college degrees all require math, and you can bet that most of those deal with statistics, or with probability: not calculus. Think about it this way: you can be an engineer with calculus, but you can be a financial speculator with statistics. With all the data being generated these days, Data Scientists are high in demand and surprisingly low in supply. Statistics offers the power of prediction, and as Nate Silver demonstrated from his election predictions, they can be a powerful crystal ball if you have the tools to use them.

But more importantly, statistics is applicable in daily life. Every day in the paper, you read about studies linking this to that, or how much the economy grew, or how popular this candidate became in that latest poll. Whether or not you realize it, statistics are everywhere and in a much more tangible and accessible way than calculus will ever be.

And it's precisely the certainty within statistics that make them seem so sound. Someone telling you that Obama is leading by 3.4\% sounds more convincing than a talking head claiming that Obama is leading. But with this kind of power, statistics can easily be used to mislead.
You may feel inclined to buy a drug advertising that 80\% of people were cured of a disease after taking it, until you read that 80\% of people who \emph{didn't} buy the drug were also cured. You might believe that vaccines can cause autism because you read about \emph{one} boy who became autistic after inoculation, until you realize you could use the same argument to link wearing coats to traffic accidents.
Like some researchers at University of South Carolina, you might notice that people who tweet more about losing weight while on a diet lose more weight. And you might be lead to believe that tweeting about losing weight causes people to lose more weight, and consequently completely miss the obvious explanation: those who are more dedicated to losing weight will tweet about it more.

This kind of ignorance is not trivial. Public opinions are shaped by statistics; policy of all kinds is shaped by statistics. If people cannot interpret for themselves what numbers mean, then numbers are no better than single-sided talking heads who interpret the numbers for them.
Statistics taken with a healthy dose of skepticism provides a far greater basis for truth than the slippery words of politicians and broadcasters, people who value an enticing fiction over a boring reality.

I'm not saying everybody has to become a statistician. But we should realize that there's more to math than $d$'s, $y$'s, and $x$'s. It's about time statistical literacy is given the priority it deserves.


\end{mla}
\end{document}

